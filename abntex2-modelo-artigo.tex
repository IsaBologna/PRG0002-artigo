%% abtex2-modelo-artigo.tex, v-1.9.7 laurocesar
%% Copyright 2012-2018 by abnTeX2 group at http://www.abntex.net.br/ 
%%
%% This work may be distributed and/or modified under the
%% conditions of the LaTeX Project Public License, either version 1.3
%% of this license or (at your option) any later version.
%% The latest version of this license is in
%%   http://www.latex-project.org/lppl.txt
%% and version 1.3 or later is part of all distributions of LaTeX
%% version 2005/12/01 or later.
%%
%% This work has the LPPL maintenance status `maintained'.
%% 
%% The Current Maintainer of this work is the abnTeX2 team, led
%% by Lauro César Araujo. Further information are available on 
%% http://www.abntex.net.br/
%%
%% This work consists of the files abntex2-modelo-artigo.tex and
%% abntex2-modelo-references.bib
%%

% ------------------------------------------------------------------------
% ------------------------------------------------------------------------
% abnTeX2: Modelo de Artigo Acadêmico em conformidade com
% ABNT NBR 6022:2018: Informação e documentação - Artigo em publicação 
% periódica científica - Apresentação
% ------------------------------------------------------------------------
% ------------------------------------------------------------------------

\documentclass[
	% -- opções da classe memoir --
	article,			% indica que é um artigo acadêmico
	11pt,				% tamanho da fonte
	oneside,			% para impressão apenas no recto. Oposto a twoside
	a4paper,			% tamanho do papel. 
	% -- opções da classe abntex2 --
	%chapter=TITLE,		% títulos de capítulos convertidos em letras maiúsculas
	%section=TITLE,		% títulos de seções convertidos em letras maiúsculas
	%subsection=TITLE,	% títulos de subseções convertidos em letras maiúsculas
	%subsubsection=TITLE % títulos de subsubseções convertidos em letras maiúsculas
	% -- opções do pacote babel --
	english,			% idioma adicional para hifenização
	brazil,				% o último idioma é o principal do documento
	sumario=tradicional
	]{abntex2}


% ---
% PACOTES
% ---

% ---
% Pacotes fundamentais 
% ---
\usepackage{lmodern}			% Usa a fonte Latin Modern
\usepackage[T1]{fontenc}		% Selecao de codigos de fonte.
\usepackage[utf8]{inputenc}		% Codificacao do documento (conversão automática dos acentos)
\usepackage{indentfirst}		% Indenta o primeiro parágrafo de cada seção.
\usepackage{nomencl} 			% Lista de simbolos
\usepackage{color}				% Controle das cores
\usepackage{graphicx}			% Inclusão de gráficos
\usepackage{microtype} 			% para melhorias de justificação
% ---
		
% ---
% Pacotes adicionais, usados apenas no âmbito do Modelo Canônico do abnteX2
% ---
\usepackage{lipsum}				% para geração de dummy text
% ---
		
% ---
% Pacotes de citações
% ---
\usepackage[brazilian,hyperpageref]{backref}	 % Paginas com as citações na bibl
\usepackage[alf]{abntex2cite}	% Citações padrão ABNT
% ---

% ---
% Configurações do pacote backref
% Usado sem a opção hyperpageref de backref
\renewcommand{\backrefpagesname}{Citado na(s) página(s):~}
% Texto padrão antes do número das páginas
\renewcommand{\backref}{}
% Define os textos da citação
\renewcommand*{\backrefalt}[4]{
	\ifcase #1 %
		Nenhuma citação no texto.%
	\or
		Citado na página #2.%
	\else
		Citado #1 vezes nas páginas #2.%
	\fi}%
% ---

% --- Informações de dados para CAPA e FOLHA DE ROSTO ---
\titulo{Artigo científico}
% \tituloestrangeiro{article: optional foreign title}

\autor{Isabella Bologna Salomão }

\local{Brasil}
\data{Junho 2021}
% ---

% ---
% Configurações de aparência do PDF final

% alterando o aspecto da cor azul
\definecolor{blue}{RGB}{41,5,195}

% informações do PDF
\makeatletter
\hypersetup{
     	%pagebackref=true,
		pdftitle={\@title}, 
		pdfauthor={\@author},
    	pdfsubject={Modelo de artigo científico com abnTeX2},
	    pdfcreator={LaTeX with abnTeX2},
		pdfkeywords={abnt}{latex}{abntex}{abntex2}{atigo científico}, 
		colorlinks=true,       		% false: boxed links; true: colored links
    	linkcolor=blue,          	% color of internal links
    	citecolor=blue,        		% color of links to bibliography
    	filecolor=magenta,      		% color of file links
		urlcolor=blue,
		bookmarksdepth=4
}
\makeatother
% --- 

% ---
% compila o indice
% ---
\makeindex
% ---

% ---
% Altera as margens padrões
% ---
\setlrmarginsandblock{3cm}{3cm}{*}
\setulmarginsandblock{3cm}{3cm}{*}
\checkandfixthelayout
% ---

% --- 
% Espaçamentos entre linhas e parágrafos 
% --- 

% O tamanho do parágrafo é dado por:
\setlength{\parindent}{1.3cm}

% Controle do espaçamento entre um parágrafo e outro:
\setlength{\parskip}{0.2cm}  % tente também \onelineskip

% Espaçamento simples
\SingleSpacing


% ----
% Início do documento
% ----
\begin{document}

% Seleciona o idioma do documento (conforme pacotes do babel)
%\selectlanguage{english}
\selectlanguage{brazil}

% Retira espaço extra obsoleto entre as frases.
\frenchspacing 

% ----------------------------------------------------------
% ELEMENTOS PRÉ-TEXTUAIS
% ----------------------------------------------------------

% Diferentes estilos de cabeçalhos e rodapés podem ser criados usando os recursos padrões do \textsf{memoir}.

%---
%
% Se desejar escrever o artigo em duas colunas, descomente a linha abaixo
% e a linha com o texto ``FIM DE ARTIGO EM DUAS COLUNAS''.
% \twocolumn[    		% INICIO DE ARTIGO EM DUAS COLUNAS
%
%---

% A norma ABNT NBR 6022:2018 não estabelece uma margem específica a ser utilizada
% no artigo científico. Dessa maneira, caso deseje alterar as margens, utilize os
% comandos abaixo:

% \begin{verbatim}
%    \setlrmarginsandblock{3cm}{3cm}{*}
%    \setulmarginsandblock{3cm}{3cm}{*}
%    \checkandfixthelayout
% \end{verbatim}

% página de titulo principal (obrigatório)
\maketitle


% titulo em outro idioma (opcional)



% resumo em português
\begin{resumoumacoluna}


Primeiro resumo - escrito antes do texto como resumo das ideias que planejo serem descritas a seguir. 

O mercado de eletrônicos, a fim de manter o consumo de dispositivos, tem apostado em novos designs e menor durabilidade dos dispositivos, além de uma produção em larga escala que inviabiliza a manutenção e promove a substituição contínua dos dispositivos. 
Comprando novos eletrônicos, o que acontece com os antigos que se tornaram obsoletos ou que pararam de funcionar? Como é feito o descarte? E qual o impacto desse crescente volume de lixo no planeta?


 \vspace{\onelineskip}
 
 \noindent
 \textbf{Palavras-chave}: 
\end{resumoumacoluna}


% resumo em inglês
\renewcommand{\resumoname}{Abstract}
\begin{resumoumacoluna}
 \begin{otherlanguage*}{english}
   

   \vspace{\onelineskip}
 
   \noindent
   \textbf{Keywords}: latex. abntex.
 \end{otherlanguage*}  
\end{resumoumacoluna}

% ]  				% FIM DE ARTIGO EM DUAS COLUNAS
% ---

\begin{center}\smaller
\textbf{Data de submissão e aprovação}: elemento obrigatório. Indicar dia, mês e ano

\textbf{Identificação e disponibilidade}: elemento opcional. Pode ser indicado o endereço eletrônico, DOI, suportes e outras informações relativas ao acesso.
\end{center}

% ----------------------------------------------------------
% ELEMENTOS TEXTUAIS
% ----------------------------------------------------------
\textual

% ----------------------------------------------------------
% Introdução
% ----------------------------------------------------------
\section{Introdução}

Os dispositivos eletrônicos já fazem parte do nosso dia a dia, tornando-se essenciais em vários contextos, sejam os nossos smartphone e computadores pessoais, eletrodomésticos ou mesmo nos hospitais e na agricultura.  É difícil de olhar em volta e não encontar ao menos um dispositivo eletrônico. 

% ----------------------------------------------------------
% Seção de explicações
% ----------------------------------------------------------
\section{Algumas reflexões a partir das leituras}

\subsection{Pontos chave - Brainstorm}

\begin{itemize}
   \item definição de Lixo eletrônico
   \item Principais compostos
   \item impacto ambiental dos compostos
   \item Reciclagem/descarte
   \item reuso
   \item Tipos de Baterias
   \item Baterias alcalinas
   \item Baterias de Lítio
   \item Extração e reciclagem do lítio
   \item impacto ambiental do extrativismo
   \item extrativismo (industrial x tradicional)
   \item impacto econômico no Brasil
   \item Obsolescência programada
   \item consumismo
   \item Desenvolvimento eletrônico no Brasil e no mundo
   \item volume de lixo, proporção do total
   \item logística reversa
   \item catadores
\end{itemize}

\subsection{Tópicos}

Para um artigo mais longo:

\begin{itemize}
	\item O que é Lixo eletrônico
	\subitem definição 
	\subitem principal componentes
	\subitem elementos químicos presentes
	\subitem volume do lixo
	\item A geração de lixo eletrônico
	\subitem consumo capitalista
	\subitem obsolescência programada
	\subitem tempo de vida médio dos dispositivos eletrônicos
	\item Impacto Ambiental
	\subitem extrativismo industrial
	\subitem impacto dos químicos na natureza
	\item A questão econômica
	\item Descarte e alternativas
	\subitem CEDIR USP
	\subitem triagem e reuso
	\subitem reciclagem
	\subitem logística reversa
\end{itemize}

Artigo de divulgação científica, mais curto: 
\begin{itemize}
	\item O que é Lixo eletrônico
	\subitem definição 
	\subitem principal componentes
	\subitem elementos químicos presentes
	\subitem volume do lixo
	\item Impacto Ambiental
	\subitem impacto dos químicos na natureza
	\item Descarte e alternativas
	\subitem CEDIR USP
	\subitem triagem e reuso
	\subitem reciclagem
	\subitem logística reversa
\end{itemize}


\subsection{Vídeos e suas referências}

\begin{itemize}
   \item Capitalismo: um sistema de lixo | 081
   \subitem \url{https://www.youtube.com/watch?v=dxbD0pUzjP0}
   \subitem \url{https://teseonze.com.br/referencias/ep081/}
   \item E o lítio? Interesses bilionários na América Latina | 071
   \subitem \url{https://www.youtube.com/watch?v=iBUQcAL0t4Y}
   \subitem \url{https://teseonze.com.br/referencias/ep071/}
   \item Ecocídio| 040
   \subitem \url{https://www.youtube.com/watch?v=4niI0mF85ek}
   \item Você Sabia? | Na USP, você pode descartar seu lixo eletrônico - Canal da USP
   \subitem \url{https://www.youtube.com/watch?v=vcuyxEMXuRc}
   \subitem \url{https://jornal.usp.br/universidade/voce-sabia-que-pode-descartar-seu-lixo-eletronico-na-usp/}
\end{itemize}

\subsubsection{Ecocídio}

Extrativismo industrial


\section{Pesquisa - artigos}

\subsection{OBSOLESCÊNCIA PROGRAMADA E TEORIA DO DECRESCIMENTO VERSUS DIREITO AO DESENVOLVIMENTO E AO CONSUMO (SUSTENTÁVEIS)}

\subsection{Da geração de renda à inclusão digital: alternativas para o lixo eletrônico}
Matéria do Jornal da USP
\cite{alternativas-lixo-eletronico}

% ---
% Finaliza a parte no bookmark do PDF, para que se inicie o bookmark na raiz
% ---
\bookmarksetup{startatroot}% 
% ---

% ---
% Conclusão
% ---
\section{Considerações finais}


% ----------------------------------------------------------
% ELEMENTOS PÓS-TEXTUAIS
% ----------------------------------------------------------
\postextual

% ----------------------------------------------------------
% Referências bibliográficas
% ----------------------------------------------------------
\bibliography{abntex2-modelo-references}

% ----------------------------------------------------------
% Glossário
% ----------------------------------------------------------
%
% Há diversas soluções prontas para glossário em LaTeX. 
% Consulte o manual do abnTeX2 para obter sugestões.
%
%\glossary

% ----------------------------------------------------------
% Apêndices
% ----------------------------------------------------------

% ---
% Inicia os apêndices
% ---
\begin{apendicesenv}
\end{apendicesenv}
% ---

% ----------------------------------------------------------
% Anexos
% ----------------------------------------------------------
\cftinserthook{toc}{AAA}
% ---
% Inicia os anexos
% ---
%\anexos
\begin{anexosenv}
\end{anexosenv}

% ----------------------------------------------------------
% Agradecimentos
% ----------------------------------------------------------

\section*{Agradecimentos}
Texto sucinto aprovado pelo periódico em que será publicado. Último elemento pós-textual.

\end{document}
